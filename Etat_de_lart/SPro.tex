\subsection{SPro}
\label{subsec:SPro}

\textbf{La bibliothèque \textit{SPro}\footnote{\url{https://pdfs.semanticscholar.org/1bc1/47e07028cbe6d2841e825c70abb60f8e1a25.pdf}} en C} est une bibliothèque logicielle de pointe conçue pour fournir des capacités avancées de traitement du signal
aux programmeurs en C. \textbf{La bibliothèque est conçue pour être hautement efficace, rapide et facile à utiliser}, ce qui en fait un choix idéal
pour une large gamme d'applications de traitement du signal.

\textbf{La bibliothèque est construite sur une architecture moderne et modulaire} qui permet aux utilisateurs de sélectionner et d'utiliser facilement
les composants qui conviennent le mieux à leurs besoins. La bibliothèque fournit un ensemble complet d'outils et d'algorithmes pour le traitement des signaux,
y compris le filtrage, l'extraction de caractéristiques, l'analyse spectrale et plus encore.

\textbf{L'une des forces principales de la bibliothèque \textit{SPro} est son efficacité}. La bibliothèque est conçue pour minimiser l'utilisation de la mémoire et minimiser \
le temps de traitement, ce qui la rend bien adaptée pour une utilisation dans des environnements en temps réel et limités en ressources. \textbf{La bibliothèque est hautement optimisée pour
      les performances}, en utilisant des techniques avancées telles que le multithreading, les instructions SIMD et les algorithmes vectorisés.

En plus de ses avantages en termes de performance, \textbf{la bibliothèque \textit{SPro} est également très conviviale pour l'utilisateur}.
La bibliothèque fournit une API simple et intuitive qui facilite aux développeurs de démarrer rapidement et de construire des applications de
traitement du signal puissantes. La bibliothèque comprend une documentation complète et un grand nombre de programmes d'exemple, qui fournissent
un excellent point de départ pour les utilisateurs qui découvrent le traitement du signal ou la bibliothèque.\\

\textit{ SPro} permet donc de paramétriser des audios, c'est à dire en extraire les paramètres, qui peuvent être modélisés comme des vecteurs contenant les
données des audios (qui seront à la suite nécessaire pour \textit{ALIZÉ} ).
Pour cela, on utilise le programme \textit{sfbcep} qui, en lui donnant une liste d'audios, donne les fichiers paramètres (d'extension \textit{.prm}).

\paragraph*{Conclusion}
\textbf{Dans l'ensemble, la bibliothèque \textit{SPro} en C est un excellent choix} pour toutes les personnes qui ont besoin de capacités avancées de traitement du signal dans leurs applications.
Avec sa haute performance, sa facilité d'utilisation et son ensemble complet d'outils et d'algorithmes, la bibliothèque convient à une large gamme d'applications et d'utilisations.

\subsubsection*{Alternatives}

\paragraph*{\textbf{\textit{HTK}}}: \\
\textbf{\textit{HTK} (HMM Tool Kit) est un kit d'outils populaire pour le traitement du signal et de la parole.} Il fournit un ensemble complet d'algorithmes et d'outils
pour la reconnaissance de la parole, la reconnaissance de l'orateur et la synthèse de la parole, entre autres choses.

\textbf{Avantages}
\begin{itemize}
      \item \textbf{Ensemble complet d'outils et d'algorithmes pour le traitement du signal et de la parole}
      \item \textbf{Bien établi et largement utilisé} dans l'industrie et l'académie
      \item Grande et active communauté d'utilisateurs
      \item \textbf{Bonne documentation} et ressources de support.
\end{itemize}


\textbf{Inconvénients}
\begin{itemize}
      \item \textbf{Peut être difficile à utiliser pour les débutants}, car le kit d'outils a une courbe d'apprentissage raide
      \item \textbf{Peut ne pas être aussi efficace que certaines des autres bibliothèques}, car il est conçu pour une large gamme
            d'applications et non spécifiquement pour les performances
\end{itemize}

\paragraph*{\textbf{\textit{MARF}}\footnote{\url{https://marf.sourceforge.net/docs/marf/0.3.0.6/report.pdf}}}:\\


\textbf{Avantages}
\begin{itemize}
      \item \textbf{Architecture modulaire} : \textit{MARF} a une architecture modulaire qui vous permet d'ajouter, de supprimer ou de remplacer facilement des composants
            pour répondre à vos besoins spécifiques.
      \item \textbf{Traitement audio} : \textit{MARF} fournit une gamme de composants de traitement audio, y compris des techniques d'extraction de caractéristiques et des
            algorithmes d'apprentissage automatique, ce qui en fait un bon choix pour une variété de tâches de traitement audio.
      \item \textbf{Basé sur Java} : \textit{MARF} est écrit en Java, qui est un langage de programmation populaire avec une large communauté de développeurs et une richesse
            de bibliothèques et d'outils.
      \item \textbf{Flexibilité} : La bibliothèque est conçue pour être flexible et personnalisable, vous permettant de l'adapter facilement à votre cas d'utilisation particulier.
\end{itemize}


\textbf{Inconvénients}
\begin{itemize}
      \item \textbf{Complexité} : \textit{MARF} possède une architecture complexe et peut être difficile à comprendre pour les développeurs qui sont nouveaux dans le traitement audio
            et l'apprentissage automatique.
      \item \textbf{Performance} : \textit{MARF} peut ne pas être aussi rapide que d'autres bibliothèques, telles que Alizé ou \textit{SPro}, car elle a été conçue pour être
            plus axée sur la flexibilité et la modularité plutôt que sur les performances.
      \item \textbf{Communauté limitée} : \textit{MARF} a une communauté plus petite par rapport à d'autres bibliothèques de traitement audio populaires, ce qui signifie que
            vous pourriez avoir plus de difficultés à trouver des réponses à vos questions ou un support pour la bibliothèque.
\end{itemize}
