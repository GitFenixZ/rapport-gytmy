
\subsection{Reconnaissance de l'individu}
\label{subsec:Reconnaissance_individu}

L'objectif primordial de notre projet est la reconnaissance de l'individu qui parle. Comme nous l'avons vu dans la section \ref{subsec:Enregistrement}, il est
possible d'enregistrer un signal audio et de le traiter pour en extraire des informations utiles. Cependant, il est nécessaire de pouvoir reconnaître
l'individu qui parle à partir de ces informations. Pour cela, nous avons étudié plusieurs bibliothèques de reconnaissance vocale, pour finalement choisir
\textit{ALIZÉ}.


\subsubsection{ALIZÉ}
\textit{ALIZÉ}\footnote{\url{https://alize.univ-avignon.fr/} et \url{https://github.com/alize-Speaker-Recognition}} est une bibliothèque open source très
puissante de reconnaissance vocale qui permet de développer rapidement et facilement des logiciels de reconnaissance vocale de haute qualité.

Elle est écrite en C++, et est munit d'une interface tant bien de bas niveau que de haut niveau, laissant ainsi une certaine liberté à l'utilisateur.


\textit{ALIZÉ} fournit des outils nécessaires et utiles pour la conception de logiciels utilisant la reconnaissance vocale. Elle permet par exemple de traiter \
des signaux audios pour pouvoir obtenir des informations utiles. Elle comprend des algorithmes de détection de début et de fin de parole, de normalisation, de
filtrage de bruit, de reconnaissance de la parole (à partir d'un fichier ou en temps réel), ...

Sa force est qu'elle met à disposition des modèles de reconnaissance très performants, des outils de développement complets, une interface haut niveau bien
plus simple à utiliser, et elle prend en charge plusieurs langues telles que le français, l'anglais ou l'espagnol. Elle est aussi multi-plateforme,
fonctionnant ainsi sur Linux, Windows et Mac OS.

Elle est aussi performante, surtout pour la reconnaissance du français, et plus documentée que la plupart des bibliothèques de reconnaissance vocale.

Pour résumer, \textit{ALIZÉ} donne accès à des outils complets pour la reconnaissance vocale. Elle est aussi multi-plateforme, et multilingues et donne accès
à une interface haut niveau facilitant son utilisation.

Son installation est cependant assez complexe, ce qui peut occasionner des pertes de temps lors du début d'un projet, et elle donne très peu d'exemples pratiques de son utilisation.

\subsubsection*{Fonctionnement de \textit{ALIZÉ}}
\textit{ALIZÉ} est un logiciel dont le but est de faciliter la reconnaissance vocale, qui travaille sur deux niveaux distincts:
\begin{enumerate}
    \item Le premier possède les données acquises, l'entrée, et le stockage.
    \item Le second, plus haut, a accès aux utilitaires et aux algorithmes qui seront manipuler par l'utilisateur (management de listes, initialisation du model, les map, ...)
\end{enumerate}

De cette manière, \textit{ALIZÉ} gère: le data, les features,  la mixture / distribution et la statistic.

\paragraph*{Architecture} :\\

\textit{ALIZÉ} possède une architecture à plusieurs couches

La base est \textit{ALIZE-Core} qui est bas niveau, incluant les fonctions pour utiliser les Gaussian mixture, et des IO pour différents formats de fichier.

Au dessus du core, il y a \textit{LIA\_RAL}, offrant des fonctionnalités de plus hauts niveaux
Il est composé de :
\begin{itemize}
    \item \textbf{\textit{LIA\_SpkDet}}  un set d'outils pour faire toutes les demandes requises par le système d'authentification du speaker, et du model training, normalization, ...
    \item \textbf{\textit{LIA\_SpkSeg}}  qui contient des outils pour le séquençage
    \item \textbf{\textit{LIA\_Utils}}  qui contient les outils pour manipuler les différents formats de données utilisés par \textit{ALIZÉ} (GMMs, features, ...)
    \item \textbf{\textit{LIA\_SpkTools}}  qui propose des fonctions plus haut niveau, au dessus du core

\end{itemize}

\subsubsection{Alternatives}

Cependant nous avons étudié d'autres bibliothèques de reconnaissance vocale, comme \textit{Kaldi} et \textit{Julius}.

\paragraph*{\textbf{\textit{Kaldi}}\footnote{\url{https://kaldi-asr.org/} et \url{https://github.com/kaldi-asr/kaldi}}}:\\

\textbf{Avantages :}
\begin{itemize}

    \item Open source
    \item Grande variété d'outil pour la formation et l'évaluation de modèles de reconnaissance vocale.
    \item Hautes performances, elle est souvent utilisée pour les applications de reconnaissance vocale exigeantes en termes de performances.
    \item Large communauté d'utilisateurs ce qui implique de nombreuses ressources en ligne et un grand nombre de personnes qui peuvent aider à résoudre les problèmes rencontrés.
    \item Flexibilité car elle peut être utilisée pour de nombreux types de reconnaissance vocale, tels que la reconnaissance de la parole en temps réel,
          la reconnaissance de commandes vocales, la reconnaissance de mots clés, ...

\end{itemize}

\textbf{Inconvénients :}
\begin{itemize}
    \item Complexe ce qui demande plus de technique et de ressources dans ce domaine.
    \item Documentation limitée
    \item Temps de développement plus long
\end{itemize}

\paragraph*{\textbf{\textit{Julius}}\footnote{\url{https://julius.osdn.jp/en_index.php} et \url{https://github.com/julius-speech/julius}}}: \\

\textbf{Avantages :}
\begin{itemize}
    \item Open source
    \item Hautes performances
    \item Large communauté d'utilisateurs
    \item Facilité d'utilisation
    \item Support de nombreux systèmes d'exploitation
\end{itemize}

\textbf{Inconvénients :}
\begin{itemize}
    \item Documentation limitée
    \item Performances plus faibles que certaines autres bibliothèques
    \item Configuration difficile
\end{itemize}

\paragraph*{\textbf{\textit{CMU Sphinx}}\footnote{\url{https://cmusphinx.github.io/} et \url{https://github.com/cmusphinx}}}: \\

\textbf{Avantages :}
\begin{itemize}
    \item Open Source
    \item Portabilité
    \item Large communauté d'utilisateurs
    \item Faible consommation de ressources
    \item Supports plusieurs langues
    \item Documentation complète
    \item Propose des modèles de reconnaissance pré-entraînés pour de nombreuses langues
\end{itemize}

\textbf{Inconvénients :}
\begin{itemize}
    \item Performances limitées (surtout pour les environnements bruyants et pour la reconnaissance de langues plus complexes)
    \item Configuration complexe
    \item Bas niveau
\end{itemize}

\subsubsection*{Conclusion des comparaisons}

Nous avons choisi d'utiliser \textit{ALIZÉ} pour la multitude d'outils qu'elle donne. De plus, elle est souvent conseillée pour sa grande capacité de reconnaissance 
vocale du français, ainsi que pour sa documentation et sa faible complexité en comparaison à d'autres bibliothèques.



