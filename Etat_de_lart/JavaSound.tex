\subsection{Enregistrement Vocal}
\label{subsec:javaSound}

Pour notre projet nous avos décidé d'utiliser \textit{Java Sound} pour l'enregistrement vocal, cependant,
ce n'était pas notre premier choix. Nous avons également étudié \textit{ALSA}.

\subsubsection*{ALSA}

\textit{ALSA} est une abréviation de "\textbf{A}dvanced \textbf{L}inux \textbf{S}ound \textbf{A}rchitecture". C'est une bibliothèque de son
de bas niveau pour Linux qui fournit un support pour la lecture et l'enregistrement audio sur ce système d'exploitation.

En utilisant \textit{ALSA}, les développeurs peuvent créer des applications audio puissantes et fiables pour \textbf{les utilisateurs de Linux},
offrant une expérience audio de qualité supérieure.\\

\subsubsection*{JavaSound}

\textit{JavaSound} est une bibliothèque de son de haut niveau pour Java. Elle fournit un support pour la lecture et l'enregistrement audio sur
les systèmes d'exploitation \textbf{Windows, Mac OS X et Linux}. Elle est fournie avec le JDK de Java.

\subsubsection*{Comparaison}
La librairie \textit{ALSA} est \textbf{rédigée en C} contrairement à \textbf{JavaSound} qui est une bibliothèque
\textbf{faite pour Java} et capable d'être intégrer sur plusieurs systèmes d'exploitation.

\textit{ALSA} peut être difficile à prendre en main malgré un panel et une diversité d'options de configuration bien supérieure à celle de \textit{JavaSound}.
Toutefois, nous sommes \textbf{limité} à une programmation \textbf{exclusivement via Java}. De plus, \textit{JavaSound} nous offre \textbf{parfaitement ce
    dont nous avons besoin} et cela à un coup réduit, notamment par le fait que cette dernière est, elle, \textbf{de haut niveau.}

D'un point de vue global, les deux bibliothèques se valent et peuvent être intéressantes dans des cadres différents.
Le choix revient donc aux programmeurs de choisir en fonction de leurs besoins et de leurs envies.
