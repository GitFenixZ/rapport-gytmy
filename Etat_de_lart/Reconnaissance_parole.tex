\subsection{Reconnaissance de la parole}
\label{subsec:Reconnaissance_parole}

Une fois que nous avons réussi à reconnaître l'individu, l'objectif est de
savoir quelle est la commande que celui-ci a donné. Pour cela, nous utilisons la
reconnaissance de la parole grâce à la bibliothèque \textit{Whisper}, plus
précisément de sa version optimisée, \textit{FasterWhisper}.

\subsubsection*{Whisper et FasterWhisper}

\textit{Whisper}, \textbf{un modèle de reconnaissance vocale} de pointe développé par OpenAI, est formé sur un vaste ensemble de données audio diversifiées. \textbf{Polyvalent}, il est
capable de gérer plusieurs tâches telles que la reconnaissance vocale multilingue, la traduction automatique de la parole et l'identification de la langue.
Cependant, l'utilisation de \textit{Whisper} présente une contrainte majeure : \textbf{son temps de traitement, qui peut aller de 10 à 15 secondes sur une machine locale}.
Pour résoudre ce problème, \textit{FasterWhisper} a été développé en utilisant
\textit{CTranslate2}, une bibliothèque d'optimisation pour les modèles de
traduction automatique. Grâce à cette optimisation, le temps de
	traitement est réduit \textbf{environ de 1 à 3 secondes}, rendant la reconnaissance vocale \textbf{quasi-instantanée}.
\textit{FasterWhisper} est également \textbf{doté d'un CLI (Command-Line Interface)}, qui facilite l'interaction avec le modèle en permettant aux utilisateurs d'exécuter des commandes directement depuis
l'invite de commande.

\subsubsection*{Avantages et limites}

Le modèle \textit{Whisper} offre de nombreux atouts. Il est \textbf{extrêmement
	polyvalent}, applicable à un large éventail de cas d'utilisation. De plus, il
\textbf{fonctionne sur une grande variété d'appareils}. Cependant, ce modèle de
reconnaissance vocale présente aussi des limites. L'une des principales
contraintes est sa \textbf{consommation importante de puissance de traitement}
pour un fonctionnement optimal, ce qui peut restreindre son utilisation sur des
appareils aux capacités de traitement limitées. Par ailleurs, même si
\textit{Whisper} peut opérer en mode hors ligne, il requiert tout de même une
connexion Internet pour la traduction de la parole et l'identification de la
langue (ce qui n'est pas utilisé dans notre projet).

\subsubsection*{Alternatives}
\textit{Whisper} peut être employé dans diverses applications, telles que les assistants virtuels multilingues, la traduction de la parole et la transcription de conférences.
Cependant, il existe \textbf{plusieurs alternatives à ce modèle, comme \textit{Google Speech-to-Text}, \textit{Amazon Transcribe} et \textit{Microsoft Azure Speech Services}}.
Chacun de ces modèles a ses avantages et ses inconvénients, et le choix dépendra des besoins spécifiques de chaque cas d'utilisation. Toutefois, contrairement à \textit{Whisper}, ces alternatives \textbf{nécessitent} une connexion Internet pour fonctionner.

En conclusion, le modèle de reconnaissance vocale \textit{Whisper} d'OpenAI et
son optimisation \textit{FasterWhisper} offrent des avantages significatifs
pour les tâches de reconnaissance vocale pertinentes dans le cadre de notre
projet.