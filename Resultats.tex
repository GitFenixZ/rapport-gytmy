\section{Résultats}
\label{sec:resultats}

En ayant testé sur des audios de tests réalisés par nous même, disponible dans le dossier \textit{DataBaseOfWords}, un taux de discrimition de l'utilisateur d'environ 60\% pour un total d'environ 400 audios testés. On remarque que certains utilisateurs ont un taux de reconnaissance proche de 100% et d'autres qui en ont un très proche de 0.

Nous avons donc réussi a développer un jeu capable de différencier, et ce, de manière plus efficace que le hasard, les joueurs à partir de leur voix dans une partie grâce à ALIZE et d'interpréter les paroles prononcées comme des commandes grâce à Whisper, c'est-à-dire comme des
actions à exécuter. 

Ainsi, le joueur peut se déplacer dans le labyrinthe juste avec sa voix si il est bien reconnu. 

Toute la gestion des modèles est faite à l'intérieur du jeu, nous n'avons donc pas besoin de lancer un autre programme pour enregistrer la voix ou pour générer les modèles. 
Le délai de réponse est assez rapide, le joueur peut se déplacer dans le labyrinthe sans avoir à attendre généralement plus de 3 secondes après avoir prononcé une commande. 

Enfin, nous avons réussi à faire en sorte que le jeu n'ai pas besoin d'accès à internet pour fonctionner, même si certaines fonctionnalités de \textit{Whisper} en ont besoin,
et aussi à réduire la demande en ressource de l'ordinateur qui exécute le programme, permettant une certaine stabilité pour les ordinateurs pas forcément performants.
