\section{Résultats}
\label{sec:resultats}

En ayant testé sur des audios de tests réalisés par nous-mêmes, disponible dans le dossier \textit{DataBaseOfWords}, un taux de discrimination de l'utilisateur d'environ 60\% est observable, pour un total d'environ 400 audios testés. Nous avons aussi remarqué que certains utilisateurs ne sont pas du tout reconnus, alors que d'autres le sont dans la quasi-totalité des cas.

Nous avons donc réussi a développer un jeu capable de différencier, et ce, de manière plus efficace que le hasard, les joueurs à partir de leur voix dans une partie grâce à ALIZE et d'interpréter les paroles prononcées comme des commandes grâce à Whisper, c'est-à-dire comme des
actions à exécuter. 

Ainsi, le joueur peut se déplacer dans le labyrinthe juste avec sa voix s'il est bien reconnu. 

Toute la gestion des modèles est faite à l'intérieur du jeu, nous n'avons donc pas besoin de lancer un autre programme pour enregistrer la voix ou pour générer les modèles. 
Le délai de réponse est assez rapide, le joueur peut se déplacer dans le labyrinthe sans avoir à attendre généralement plus de 3 secondes après avoir prononcé une commande. 

Enfin, nous avons réussi à faire en sorte que le jeu n'ait pas besoin d'un accès à internet pour fonctionner, même si certaines fonctionnalités de \textit{Whisper} en ont besoin.
De plus, nous avons également réduit la quantité de ressources nécessaires à un ordinateur pour exécuter correctement notre programme, permettant alors une certaine stabilité pour ceux étant peu performants.
