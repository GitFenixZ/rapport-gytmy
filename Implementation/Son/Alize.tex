\subsubsection{Alizé}
\label{sec:alizeImpl}


\paragraph*{Modélisation} : \\

Afin de reconnaître un utilisateur, nous avons tout d'abord besoin de modèles.
Un modèle pour le monde qui prend les paramètres du plus d'audios possible pour enrichir son environnement.
Un modèle pour l'interlocuteur qui prend les paramètres du plus d'audios possible pour un interlocuteur.

Le modèle du monde permet de créer le modèle de l'interlocuteur, en lui donnant l'environnement dans lequel il est.
Plus les modèles sont enrichis, et plus ils seront efficaces.

Pour créer les modèles, nous avons tout d'abord besoin des paramètres fournit par \textit{SPro}.

Ensuite, nous allons utiliser deux programmes afin d'améliorer la modélisation :
\begin{itemize}
    \item\textit{\textbf{EnergyDetector}} qui prend des paramètres d'audios et qui renvoie dans un fichier label les différents moments où
    l'interlocuteur parle (pour les différencier à quand il ne parle pas), d'extension \textit{.lbl}.
    \item\textit{\textbf{NormFeat}} qui prend les labels et les paramètres des audios afin de les normaliser, et qui renvoie de
    nouveaux paramètres normalisés d'extension \textit{.norm.prm}.
\end{itemize}


On procède alors à la modélisation du monde avec tout les paramètres du monde normalisés et des labels, afin d'obtenir son modèle qui est un \textbf{GMM}
(Gaussian Mixture Model), et qui a comme extension \textit{.gmm}.

Pour cela, on utilise le programme TrainWorld de \textit{  LIA\_SpkDet}.

A l'aide du modèle du monde, et des paramètres normalisés des interlocuteurs ainsi que des labels, on crée leurs modèles, qui seront aussi des \textbf{GMM},
d'extension \textit{.gmm}.
our cela, on utilise le programme TrainTarget de \textit{ LIA\_SpkDet}.


\paragraph*{Reconnaissance} : \\

Maintenant que les modèles des interlocuteurs sont créés, on va les utiliser pour reconnaître un interlocuteur.

Pour cela, on utilise le programme ComputeScore de \textit{LIA\_SpkDet}.


Ce programme va prendre dans un fichier, les audios à comparer, suivis des interlocuteurs qu'on veut essayer de reconnaître. Il donnera à la fin dans un \textit{.txt}
toutes les données de comparaison et donc une valeur de comparaison.

Pour reconnaître un interlocuteur, on utilise toutes les valeurs données, afin de tirer l'interlocuteur qui a la valeur maximum de comparaison, et dépassant
un certain seuil (choisir arbitrairement selon les tests résultant).

Cet utilisateur sera donc celui qui est reconnu.


\paragraph*{Utilisation dans le projet} : \\
Dans notre jeu, on réalise toutes les étapes précédentes, en prenant en compte la gestion de fichiers de notre jeu. Les listes utilisées pour la modélisation,
afin de fournir les informations au programme de quels audios paramétriser, normaliser et modéliser, sont actualisés au moment où la modélisation a besoin
d'être faite (c'est-à-dire quand il y a une modification dans les audios).

Lors de cette étape, tout les audios sont paramétrisés et normalisés, le modèle du monde est créé avec tout les audios disponibles.
Les modèles des interlocuteurs (ou des mots d'interlocuteur) sont à la suite créé avec les audios séparés par utilisateur (et par mot) dans le gestionnaire.

La modélisation se fait dans fichier \textit{ src/main/exe/model} qui contient tout les répertoire séparant la modélisation.

La reconnaissance, enfin, se fait en comparant tous les modèles des interlocuteurs avec les paramètres normalisés de l'audio émis pour une commande dans la partie : currentAudio.

À la fin, si un interlocuteur qui a la valeur la plus haute de comparaison, atteint un certain seuil, ce joueur jouera la commande donnée.


\paragraph*{Intégration en Java} : \\
Afin de gérer toute la partie audio, nous avons créé une classe \textbf{\textit{ModelManager}} qui s'occupe de la modélisation et de la reconnaissance. Elle
peut exécuter les programmes de \textit{ALIZÉ} et de \textit{SPro} grâce à la classe \textbf{\textit{RunSH}}. L'objectif de \textbf{\textit{ModelManager}}
est, non seulement l'exécution des programmes, mais aussi la gestion des fichiers et des listes de fichiers à utiliser pour la modélisation et la
reconnaissance.
Ces fonctions sont partagées avec la classe \textbf{\textit{AudioManager}} qui s'occupe de la partie audio.

Comment obtenir un résultat? Une fois que \textit{ALIZÉ} a fait son travail il faut interpréter le résultat. Ce procédé est réalisé par les classes
\textbf{\textit{AlizeRecognitionResultParse}}, qui s'occupe purement de parser le résultat de la reconnaissance, et \textbf{\textit{AudioRecognitionResult}}
qui s'occupe de la gestion des résultats de reconnaissance.


%TODO: insert more information and more details about the implementation of the audio recognition