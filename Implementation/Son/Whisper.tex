\subsubsection{Whisper}
\label{sec:whisperImpl}

\paragraph*{Utilisation du client FasterWhisper} : \\
Suite à l'installation de \textit{FasterWhisper}, un CLI (interface en ligne de
commande) est accessible depuis le terminal, servant d'interface avec le modèle
\textit{FasterWhisper} optimisé à partir du modèle \textit{Whisper} d'OpenAI.
Pour optimiser les paramètres du modèle et pour faciliter l'utilsation du CLI,
un script bash nommé \textbf{whisper.sh} a été implémenté.

En faisant appel au CLI de \textit{FasterWhisper} sur un fichier audio, un fichier JSON est généré, contenant les
informations cruciales extraites du fichier audio.

\paragraph*{Intégration avec Java}: \\
L'application accède aux fonctionnalités de \textit{FasterWhisper} grâce à la
classe \textbf{Whisper.java}, qui prend en charge l'exécution du script bash et
la récupération du résultat de la transcription. Une fois le processus effectué
par \textit{FasterWhisper}, le texte transcrit est récupéré à l'aide d'un
parseur JSON provenant de la bibliothèque Jackson, préalablement intégrée au
projet. Ce parseur exploite le modèle \textbf{WhisperResult.java}, faisant
office de structure pour le fichier JSON, pour atteindre son objectif. Tout
cela se fait de manière \textit{Thread-Safe}, pour éviter les problèmes de
concurrence. La classe \textbf{ThreadedQueue.java} est utilisée pour gérer un
système de file d'attente, permettant de gérer les requêtes concurrentes.