\subsubsection{Whisper}
\label{sec:whisperImpl}

\paragraph*{Utilisation du client Whisper} : \\

Suite à l'installation de \textit{Whisper}, un client est accessible depuis le terminal, faisant office d'interface avec le modèle \textit{Whisper} d'OpenAI. 
Pour faciliter son utilisation et optimiser les paramètres, un script bash nommé \textbf{whisper.sh} a été mis en place.

En utilisant \textit{Whisper} sur un fichier audio et en fonction des paramètres prédéfinis, un fichier JSON est généré, contenant les informations cruciales extraites du fichier audio d'origine.

\paragraph*{Intégration avec Java} : \\

L'application accède aux fonctionnalités de \textit{Whisper} grâce à la classe \textbf{Whisper.java}, qui prend en charge l'exécution du script bash et la récupération du résultat de la transcription. 
Une fois le processus effectué par \textit{Whisper}, le texte transcrit est récupéré à l'aide d'un parseur JSON provenant de la bibliothèque Jackson, préalablement intégrée au projet. 
Ce parseur exploite le modèle \textbf{WhisperResult.java}, faisant office de structure pour le fichier JSON, pour atteindre son objectif.