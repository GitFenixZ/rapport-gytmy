\subsubsection{Whisper}
\label{sec:whisperImpl}

\paragraph*{Utilisation du client FasterWhisper} : \\ Après l'installation de
\textit{FasterWhisper}, un interface en ligne de commande (CLI) est disponible
depuis le terminal. Cette interface permet d'interagir avec le modèle
\textit{FasterWhisper}, qui a été amélioré à partir du modèle \textit{Whisper}
d'OpenAI. Afin de configurer les paramètres du modèle et de simplifier
l'utilisation de l'interface en ligne de commande, un script bash appelé
\textbf{whisper.sh} a été créé.

En faisant appel au CLI de \textit{FasterWhisper} sur un fichier audio, un
fichier JSON est généré, contenant les informations cruciales extraites du
fichier audio.

\paragraph*{Intégration avec Java}: \\ L'application accède aux fonctionnalités de
\textit{FasterWhisper} grâce à la classe \textbf{Whisper}, qui prend en
charge l'exécution du script bash et la récupération du résultat de la
transcription. Une fois le processus effectué par \textit{FasterWhisper}, le
texte transcrit est récupéré à l'aide d'un parseur JSON provenant de la
bibliothèque Jackson, préalablement intégrée au projet. Ce parseur exploite le
modèle \textbf{WhisperResult}, faisant office de structure pour le fichier
JSON, pour atteindre son objectif. Tout cela se fait de manière
\textit{Thread-Safe}, pour éviter les problèmes de concurrence. La classe
\textbf{ThreadedQueue} est utilisée pour gérer un système de file
d'attente, permettant de gérer les requêtes concurrentes.