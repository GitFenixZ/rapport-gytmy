\subsubsection{Le Modèle}
\label{subsubsec:modele}

Nous avons une interface \textbf{\textit{LabyrinthModel.java}} avec son implémentation
\textbf{\textit{LabyrinthModelImplementation.java}} qui servent à modéliser le Labyrinthe.
Afin de représenter le labyrinthe, nous avons utilisé une matrice de booléens
où '\textit{true}' représente un chemin et '\textit{false}' représente un mur.
% TODO: Ajouter une image du diagramme des classes de l'implémentation de l'interface \textbf{\textit{LabyrinthModel.java}}' et de son implémentation '\textbf{\textit{LabyrinthModelImplementation.java}}

Les labyrinthes sont générés grâce à un générateur qui est géré par
l'interface \textbf{\textit{BoardGenerator.java}}' et son implémentation '\textbf{\textit{DepthFirstGenerator.java}}.
L'algorithme de génération utilisé est le Depth-First Search.
% TODO: Ajouter une image du diagramme des classes de l'implémentation de l'interface \textbf{\textit{BoardGenerator.java}}' et de son implémentation '\textbf{\textit{DepthFirstGenerator.java}}

L'enum \textbf{\textit{GameMode.java}} permet de définir les différents modes de jeu.
Chaque type de jeu a sa propre implémentation de l'interface \textbf{\textit{GameModeData.java}}
qui sert à définir les données des paramètres de jeu
(taille, type de score, difficulté) et la classe \textbf{\textit{GameData.java}} se charge de
regrouper ces données avec la liste des joueurs.
% TODO: Ajouter une image du diagramme des classes de l'implémentation de l'enum \textbf{\textit{GameMode.java}}' avec son implémentation '\textbf{\textit{GameModeData.java}}' et de la case '\textbf{\textit{GameData.java}}

Les labyrinthes sont créés grâce aux méthodes de la classe
\textbf{\textit{LabyrinthModelFactory.java}} qui se charge de mettre à disposition des méthodes
qui permettent de créer des labyrinthes de différentes tailles selon les modes
de jeu.
% TODO: Ajouter une image du diagramme des classes de l'implémentation de la classe \textbf{\textit{LabyrinthModelFactory.java}}

En ce qui concerne les joueurs, nous avons une interface \textbf{\textit{Player.java}} avec son
implémentation \textbf{\textit{PlayerImplementation.java}} qui permettent de représenter un
joueur. Ces derniers sont caractérisés par un nom, une couleur, des coordonnées, un
statut, un nombre de mouvements et un temps qui correspond au temps pris pour finir le labyrinthe.
Ces deux dernières données servent à calculer le score d'un joueur.
%TODO : Ajouter une image du diagramme des classes de l'implémentation de l'interface \textbf{\textit{Player.java}}' et de son implémentation '\textbf{\textit{PlayerImplementation.java}}

Pour pouvoir calculer le score des joueurs nous avons plusieurs éléments qui
utilisent les données évoquées précédemment. L'interface \textbf{\textit{ScoreCalculator.java}}
et ses implémentations permettent de calculer le score d'un joueur à partir
d'une implémentation de \textbf{\textit{ScoreInfo.java}} qui stocke les informations
nécessaires. La classe \textbf{\textit{ScoreCalculatorFactory.java}} se charge de créer des
instances d'implmémentaions de \textbf{\textit{ScoreCalculator.java}} en fonction du type de
score et des informations selon le type de score (respectivement grâce à l'enum
\textbf{\textit{ScoreType.java}}, et ScoreInfo.java). Le calcul du score se fait
mathématiquement en fonction du nombre de mouvements, du temps passé et de la
difficulté du labyrinthe.
% TODO : Ajouter une image du diagramme des classes de \textbf{\textit{ScoreCalculator.java}}' et de ses implémentations, de '\textbf{\textit{ScoreCalculatorFactory.java}}', de l'enum '\textbf{\textit{ScoreType.java}}' et de '\textbf{\textit{ScoreInfo.java}}

L'enum \textbf{\textit{Direction.java}} permet de définir les quatre directions de déplacements.
% TODO : Ajouter une image du diagramme des classes de l'enum \textbf{\textit{Direction.java}}