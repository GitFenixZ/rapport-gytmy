\subsection{Le labyrinthe}
\label{subsec:labyrinthe}

% Pour chaque partie de l'implémentation du labyrinthe je pense présenter les fichiers .java (classes, interfaces, enums...) qui gèrent chaque secction de l'implémentation. En accompagnant chaque section d'une image de la partie pertinente du diagramme de classe.

\subsubsection{Le Modèle}

Nous avons une interface 'LabyrinthModel.java' et son implémentation 'LabyrinthModelImplementation.java' qui servent à modéliser le Labyrinthe. Afin de représenter le labyrinthe, nous avons utilisé une matrice de booléens où 'true' représente un chemin et 'false' représente un mur. Les labyrinthes sont générés grâces à un générateur qui est géré par l'interface 'BoardGenerator.java' et son implémentation 'DepthFirstGenerator.java'. L'algorithme de génération utilisé est le Depth-First Search.

\subsubsection{La vue}

\subsubsection{Le Contrôleur}







