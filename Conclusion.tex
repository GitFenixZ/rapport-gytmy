\section{Conclusion}
\label{sec:conclusion}

Ce projet a été très enrichissant car il nous a permis, non seulement d'approfondir nos connaissances en Java, mais aussi d'apprendre à rechercher des
informations sur des librairies inconnues. De plus, nous avons beaucoup apprécié le fait que le sujet porte sur la reconnaissance vocale et la gestion et
manipulation de fichiers audio, ce qui n'est pas souvent abordé dans les cours de programmation.

Cependant, nous avons eu des difficultés dues à la complexité de certaines librairies, en particulier ALIZÉ. Même si, dès le début, on a mis l'accent sur la
recherche, nous avons eu du mal à trouver des informations sur les libraries que nous avons utilisées, surtout à cause de l'ancienneté de leurs mises-à-jour.
Du point de vue de l'implémentation du jeu tout s'est bien passé. Malgré cela, le fait de ne pas maîtriser les librairies nous a fait perdre du temps, ce qui
a eu un impact sur la qualité du jeu car on n'a pas pu passer autant de temps que nous l'aurions souhaité sur l'aspect graphique. Toutefois, nous sommes
très satisfaits du résultat final car nous avons réussi à implémenter un jeu de qualité, qui est très agréable à jouer et qui n'a pas besoin d'autres
applications pour fonctionner: une fois installé tout se passe via l'interface graphique.

